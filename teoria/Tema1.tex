% Options for packages loaded elsewhere
\PassOptionsToPackage{unicode}{hyperref}
\PassOptionsToPackage{hyphens}{url}
%
\documentclass[
  ignorenonframetext,
]{beamer}
\usepackage{pgfpages}
\setbeamertemplate{caption}[numbered]
\setbeamertemplate{caption label separator}{: }
\setbeamercolor{caption name}{fg=normal text.fg}
\beamertemplatenavigationsymbolsempty
% Prevent slide breaks in the middle of a paragraph
\widowpenalties 1 10000
\raggedbottom
\setbeamertemplate{part page}{
  \centering
  \begin{beamercolorbox}[sep=16pt,center]{part title}
    \usebeamerfont{part title}\insertpart\par
  \end{beamercolorbox}
}
\setbeamertemplate{section page}{
  \centering
  \begin{beamercolorbox}[sep=12pt,center]{part title}
    \usebeamerfont{section title}\insertsection\par
  \end{beamercolorbox}
}
\setbeamertemplate{subsection page}{
  \centering
  \begin{beamercolorbox}[sep=8pt,center]{part title}
    \usebeamerfont{subsection title}\insertsubsection\par
  \end{beamercolorbox}
}
\AtBeginPart{
  \frame{\partpage}
}
\AtBeginSection{
  \ifbibliography
  \else
    \frame{\sectionpage}
  \fi
}
\AtBeginSubsection{
  \frame{\subsectionpage}
}
\usepackage{amsmath,amssymb}
\usepackage{lmodern}
\usepackage{iftex}
\ifPDFTeX
  \usepackage[T1]{fontenc}
  \usepackage[utf8]{inputenc}
  \usepackage{textcomp} % provide euro and other symbols
\else % if luatex or xetex
  \usepackage{unicode-math}
  \defaultfontfeatures{Scale=MatchLowercase}
  \defaultfontfeatures[\rmfamily]{Ligatures=TeX,Scale=1}
\fi
% Use upquote if available, for straight quotes in verbatim environments
\IfFileExists{upquote.sty}{\usepackage{upquote}}{}
\IfFileExists{microtype.sty}{% use microtype if available
  \usepackage[]{microtype}
  \UseMicrotypeSet[protrusion]{basicmath} % disable protrusion for tt fonts
}{}
\makeatletter
\@ifundefined{KOMAClassName}{% if non-KOMA class
  \IfFileExists{parskip.sty}{%
    \usepackage{parskip}
  }{% else
    \setlength{\parindent}{0pt}
    \setlength{\parskip}{6pt plus 2pt minus 1pt}}
}{% if KOMA class
  \KOMAoptions{parskip=half}}
\makeatother
\usepackage{xcolor}
\newif\ifbibliography
\usepackage{color}
\usepackage{fancyvrb}
\newcommand{\VerbBar}{|}
\newcommand{\VERB}{\Verb[commandchars=\\\{\}]}
\DefineVerbatimEnvironment{Highlighting}{Verbatim}{commandchars=\\\{\}}
% Add ',fontsize=\small' for more characters per line
\usepackage{framed}
\definecolor{shadecolor}{RGB}{248,248,248}
\newenvironment{Shaded}{\begin{snugshade}}{\end{snugshade}}
\newcommand{\AlertTok}[1]{\textcolor[rgb]{0.94,0.16,0.16}{#1}}
\newcommand{\AnnotationTok}[1]{\textcolor[rgb]{0.56,0.35,0.01}{\textbf{\textit{#1}}}}
\newcommand{\AttributeTok}[1]{\textcolor[rgb]{0.77,0.63,0.00}{#1}}
\newcommand{\BaseNTok}[1]{\textcolor[rgb]{0.00,0.00,0.81}{#1}}
\newcommand{\BuiltInTok}[1]{#1}
\newcommand{\CharTok}[1]{\textcolor[rgb]{0.31,0.60,0.02}{#1}}
\newcommand{\CommentTok}[1]{\textcolor[rgb]{0.56,0.35,0.01}{\textit{#1}}}
\newcommand{\CommentVarTok}[1]{\textcolor[rgb]{0.56,0.35,0.01}{\textbf{\textit{#1}}}}
\newcommand{\ConstantTok}[1]{\textcolor[rgb]{0.00,0.00,0.00}{#1}}
\newcommand{\ControlFlowTok}[1]{\textcolor[rgb]{0.13,0.29,0.53}{\textbf{#1}}}
\newcommand{\DataTypeTok}[1]{\textcolor[rgb]{0.13,0.29,0.53}{#1}}
\newcommand{\DecValTok}[1]{\textcolor[rgb]{0.00,0.00,0.81}{#1}}
\newcommand{\DocumentationTok}[1]{\textcolor[rgb]{0.56,0.35,0.01}{\textbf{\textit{#1}}}}
\newcommand{\ErrorTok}[1]{\textcolor[rgb]{0.64,0.00,0.00}{\textbf{#1}}}
\newcommand{\ExtensionTok}[1]{#1}
\newcommand{\FloatTok}[1]{\textcolor[rgb]{0.00,0.00,0.81}{#1}}
\newcommand{\FunctionTok}[1]{\textcolor[rgb]{0.00,0.00,0.00}{#1}}
\newcommand{\ImportTok}[1]{#1}
\newcommand{\InformationTok}[1]{\textcolor[rgb]{0.56,0.35,0.01}{\textbf{\textit{#1}}}}
\newcommand{\KeywordTok}[1]{\textcolor[rgb]{0.13,0.29,0.53}{\textbf{#1}}}
\newcommand{\NormalTok}[1]{#1}
\newcommand{\OperatorTok}[1]{\textcolor[rgb]{0.81,0.36,0.00}{\textbf{#1}}}
\newcommand{\OtherTok}[1]{\textcolor[rgb]{0.56,0.35,0.01}{#1}}
\newcommand{\PreprocessorTok}[1]{\textcolor[rgb]{0.56,0.35,0.01}{\textit{#1}}}
\newcommand{\RegionMarkerTok}[1]{#1}
\newcommand{\SpecialCharTok}[1]{\textcolor[rgb]{0.00,0.00,0.00}{#1}}
\newcommand{\SpecialStringTok}[1]{\textcolor[rgb]{0.31,0.60,0.02}{#1}}
\newcommand{\StringTok}[1]{\textcolor[rgb]{0.31,0.60,0.02}{#1}}
\newcommand{\VariableTok}[1]{\textcolor[rgb]{0.00,0.00,0.00}{#1}}
\newcommand{\VerbatimStringTok}[1]{\textcolor[rgb]{0.31,0.60,0.02}{#1}}
\newcommand{\WarningTok}[1]{\textcolor[rgb]{0.56,0.35,0.01}{\textbf{\textit{#1}}}}
\usepackage{longtable,booktabs,array}
\usepackage{calc} % for calculating minipage widths
\usepackage{caption}
% Make caption package work with longtable
\makeatletter
\def\fnum@table{\tablename~\thetable}
\makeatother
\usepackage{graphicx}
\makeatletter
\def\maxwidth{\ifdim\Gin@nat@width>\linewidth\linewidth\else\Gin@nat@width\fi}
\def\maxheight{\ifdim\Gin@nat@height>\textheight\textheight\else\Gin@nat@height\fi}
\makeatother
% Scale images if necessary, so that they will not overflow the page
% margins by default, and it is still possible to overwrite the defaults
% using explicit options in \includegraphics[width, height, ...]{}
\setkeys{Gin}{width=\maxwidth,height=\maxheight,keepaspectratio}
% Set default figure placement to htbp
\makeatletter
\def\fps@figure{htbp}
\makeatother
\setlength{\emergencystretch}{3em} % prevent overfull lines
\providecommand{\tightlist}{%
  \setlength{\itemsep}{0pt}\setlength{\parskip}{0pt}}
\setcounter{secnumdepth}{-\maxdimen} % remove section numbering
\ifLuaTeX
  \usepackage{selnolig}  % disable illegal ligatures
\fi
\IfFileExists{bookmark.sty}{\usepackage{bookmark}}{\usepackage{hyperref}}
\IfFileExists{xurl.sty}{\usepackage{xurl}}{} % add URL line breaks if available
\urlstyle{same} % disable monospaced font for URLs
\hypersetup{
  pdftitle={Tema 1 - Trabajando con R},
  pdfauthor={Juan Gabriel Gomila \& María Santos},
  hidelinks,
  pdfcreator={LaTeX via pandoc}}

\title{Tema 1 - Trabajando con R}
\author{Juan Gabriel Gomila \& María Santos}
\date{}

\begin{document}
\frame{\titlepage}

\hypertarget{conociendo-r}{%
\section{Conociendo R}\label{conociendo-r}}

\begin{frame}{¿Qué es R?}
\protect\hypertarget{quuxe9-es-r}{}
\includegraphics{Imgs/RLogo.jpg}

\begin{itemize}
\tightlist
\item
  Entorno de programación para el análisis estadístico y gráfico de
  datos
\item
  Software libre
\item
  Sintaxis sencilla e intuitiva
\item
  Enorme comunidad de usuarios (Comprehensive R Archive Network, CRAN)
\item
  ¿Aún tenéis dudas de por qué usarlo?
  \href{https://www.r-bloggers.com/why-use-r-five-reasons/}{Haz click
  aquí}
\end{itemize}
\end{frame}

\begin{frame}{¿Qué es RStudio?}
\protect\hypertarget{quuxe9-es-rstudio}{}
En este curso usaremos RStudio como interfaz gráfica de usuario de R
para todos los sistemas operativos

Es un entorno integrado para utilizar y programar con R

\includegraphics{Imgs/InterfazRStudio.png}
\end{frame}

\begin{frame}[fragile]{Cómo instalar R}
\protect\hypertarget{cuxf3mo-instalar-r}{}
\textbf{Si sois de Windows o Mac}

\begin{enumerate}
\tightlist
\item
  Id a \href{http://cran.r-project.org/}{CRAN}
\item
  Pulsad sobre el enlace correspondiente a vuestro sistema operativo
\item
  Seguid las instrucciones de instalación correspondientes
\end{enumerate}

\textbf{Si trabajáis con Ubuntu o Debian}

\begin{enumerate}
\tightlist
\item
  Abrid la terminal, estando conectados a internet
\item
  Introducid lo siguiente: \texttt{sudo\ aptitude\ install\ r-base}
\end{enumerate}
\end{frame}

\begin{frame}[fragile]{Cómo instalar RStudio}
\protect\hypertarget{cuxf3mo-instalar-rstudio}{}
\begin{enumerate}
\tightlist
\item
  \href{http://www.rstudio.com/products/rstudio/download/}{Obtener
  RStudio}
\item
  \textbf{Solo si utilizáis Linux}, ejecutad en una terminal la
  siguiente instrucción para completar la instalación:
  \texttt{sudo\ dpkg\ -i\ rstudio-\textless{}version\textgreater{}-i386.deb},
  donde \texttt{version} refiere a la versión concreta que se haya
  descargado
\end{enumerate}

\includegraphics{Imgs/RSLogo.jpg}
\end{frame}

\begin{frame}{Trabajando con RStudio}
\protect\hypertarget{trabajando-con-rstudio}{}
\includegraphics{Imgs/Disquete.png} \includegraphics{Imgs/Carpeta.png}
\includegraphics{Imgs/RScript.png} \includegraphics{Imgs/RNotebook.png}
\includegraphics{Imgs/RMarkdown.png} \includegraphics{Imgs/Shiny.png}
\includegraphics{Imgs/TextFile.png} \includegraphics{Imgs/C++File.png}
\includegraphics{Imgs/RSweave.png} \includegraphics{Imgs/RHTML.png}
\includegraphics{Imgs/RPresentation.png}
\includegraphics{Imgs/RDocumentation.png}

\includegraphics{Imgs/Easy.jpg}
\end{frame}

\begin{frame}[fragile]{Cómo pedir ayuda}
\protect\hypertarget{cuxf3mo-pedir-ayuda}{}
\begin{itemize}
\tightlist
\item
  \texttt{help()}: obtener ayuda por consola
\item
  \texttt{??...}: obtener ayuda por consola
\item
  Pestaña \texttt{Help} de Rstudio
\item
  \href{https://www.rstudio.com/wp-content/uploads/2015/02/rmarkdown-cheatsheet.pdf}{Cheat
  Sheet de RStudio}
\item
  Buscar en San Google (stackoverflow, R project\ldots)
\item
  Foro del curso
\end{itemize}

\includegraphics{Imgs/help.png}
\end{frame}

\begin{frame}[fragile]{Paquetes: cómo instalarlos y cargarlos}
\protect\hypertarget{paquetes-cuxf3mo-instalarlos-y-cargarlos}{}
Paquete. Librería con funciones y datos que no necesariamente vienen
instaladas de serie

\begin{itemize}
\tightlist
\item
  \texttt{install.packages("nombre\_paquete",\ dep\ =\ TRUE)}: instala o
  actualiza un paquete de R
\item
  \texttt{library(nombre\_del\_paquete)}: carga un paquete ya instalado
\end{itemize}
\end{frame}

\hypertarget{utilizando-r-como-calculadora}{%
\section{Utilizando R como
calculadora}\label{utilizando-r-como-calculadora}}

\begin{frame}[fragile]{Calculadora básica - Operaciones}
\protect\hypertarget{calculadora-buxe1sica---operaciones}{}
\begin{longtable}[]{@{}ll@{}}
\toprule()
Código & Operación \\
\midrule()
\endhead
\texttt{+} & Suma \\
\texttt{-} & Resta \\
\texttt{*} & Multiplicación \\
\texttt{/} & División \\
\texttt{\^{}} & Potencia \\
\texttt{\%/\%} & Cociente entero \\
\texttt{\%\%} & Resto de división entera \\
\bottomrule()
\end{longtable}
\end{frame}

\begin{frame}[fragile]{Calculadora básica - Operaciones}
\protect\hypertarget{calculadora-buxe1sica---operaciones-1}{}
\begin{longtable}[]{@{}ll@{}}
\toprule()
Código & Significado \\
\midrule()
\endhead
\texttt{pi} & \href{https://es.wikipedia.org/wiki/Número_π}{\(\pi\)} \\
\texttt{Inf} &
\href{https://es.wikipedia.org/wiki/Infinito}{\(\infty\)} \\
\texttt{NaN} & Indeterminación (Not a Number) \\
\texttt{NA} & Valor desconocido (Not Available) \\
\bottomrule()
\end{longtable}
\end{frame}

\begin{frame}[fragile]{Calculadora básica - Operaciones}
\protect\hypertarget{calculadora-buxe1sica---operaciones-2}{}
\begin{Shaded}
\begin{Highlighting}[]
\DecValTok{2}\SpecialCharTok{+}\DecValTok{2}
\end{Highlighting}
\end{Shaded}

\begin{verbatim}
[1] 4
\end{verbatim}

\begin{Shaded}
\begin{Highlighting}[]
\DecValTok{77}\SpecialCharTok{\%/\%}\DecValTok{5}
\end{Highlighting}
\end{Shaded}

\begin{verbatim}
[1] 15
\end{verbatim}

\begin{Shaded}
\begin{Highlighting}[]
\DecValTok{77}\SpecialCharTok{\%\%}\DecValTok{5}
\end{Highlighting}
\end{Shaded}

\begin{verbatim}
[1] 2
\end{verbatim}
\end{frame}

\begin{frame}[fragile]{Calculadora básica - Funciones}
\protect\hypertarget{calculadora-buxe1sica---funciones}{}
\begin{longtable}[]{@{}ll@{}}
\toprule()
Código & Función \\
\midrule()
\endhead
\texttt{sqrt(x)} & \(\sqrt{x}\) \\
\texttt{exp(x)} & \(e^x\) \\
\texttt{log(x)} & \(\ln(x)\) \\
\texttt{log10(x)} & \(\log_{10}(x)\) \\
\texttt{log(x,a)} & \(\log_a(x)\) \\
\texttt{abs(x)} & \(\begin{vmatrix}x\end{vmatrix}\) \\
\bottomrule()
\end{longtable}
\end{frame}

\begin{frame}[fragile]{Calculadora básica - Funciones}
\protect\hypertarget{calculadora-buxe1sica---funciones-1}{}
\begin{Shaded}
\begin{Highlighting}[]
\FunctionTok{sqrt}\NormalTok{(}\DecValTok{9}\NormalTok{)}
\end{Highlighting}
\end{Shaded}

\begin{verbatim}
[1] 3
\end{verbatim}

\begin{Shaded}
\begin{Highlighting}[]
\FunctionTok{log}\NormalTok{(}\FunctionTok{exp}\NormalTok{(}\DecValTok{1}\NormalTok{))}
\end{Highlighting}
\end{Shaded}

\begin{verbatim}
[1] 1
\end{verbatim}

\begin{Shaded}
\begin{Highlighting}[]
\FunctionTok{log}\NormalTok{(}\DecValTok{1000}\NormalTok{,}\DecValTok{10}\NormalTok{)}
\end{Highlighting}
\end{Shaded}

\begin{verbatim}
[1] 3
\end{verbatim}

\begin{Shaded}
\begin{Highlighting}[]
\FunctionTok{log10}\NormalTok{(}\DecValTok{1000}\NormalTok{)}
\end{Highlighting}
\end{Shaded}

\begin{verbatim}
[1] 3
\end{verbatim}
\end{frame}

\begin{frame}[fragile]{Calculadora básica - Combinatoria}
\protect\hypertarget{calculadora-buxe1sica---combinatoria}{}
\begin{longtable}[]{@{}ll@{}}
\toprule()
Código & Operación \\
\midrule()
\endhead
\texttt{factorial(x)} &
\href{https://es.wikipedia.org/wiki/Factorial}{\(x!\)} \\
\texttt{choose(n,m)} & \(\begin{pmatrix}n\\ m\end{pmatrix}\) \\
\bottomrule()
\end{longtable}

\vspace{0.2cm}

\begin{itemize}
\tightlist
\item
  Número factorial. Se define como número factorial de un número entero
  positivo \(n\) como \(n!=n\cdot(n-1)\cdots 2\cdot 1\)
\item
  \href{https://es.wikipedia.org/wiki/Coeficiente_binomial}{Coeficiente
  binomial}. Se define el coeficiente binomial de \(n\) sobre \(m\) como
  \[\begin{pmatrix}n\\ m\end{pmatrix}=\frac{n!}{m!(n-m)!}\]
\end{itemize}
\end{frame}

\begin{frame}{Calculadora básica - Combinatoria}
\protect\hypertarget{calculadora-buxe1sica---combinatoria-1}{}
\href{https://es.wikipedia.org/wiki/Triángulo_de_Pascal}{Triángulo de
Pascal}.

\usepackage{mathdots}
\usepackage{yhmath}
\usepackage{mathdots}
\usepackage{MnSymbol}

\[\begin{matrix}
&&&&&1&&&&&\\
&&&&1&&1&&&&\\
&&&1&&2&&1&&&\\
&&1&&3&&3&&1&&\\
&1&&4&&6&&4&&1&\\
1&&5&&10&&10&&5&&1\end{matrix}\]

que se corresponde con \ldots{}
\end{frame}

\begin{frame}{Calculadora básica - Combinatoria}
\protect\hypertarget{calculadora-buxe1sica---combinatoria-2}{}
\[\begin{matrix}
&&&&\begin{pmatrix}0\\0\end{pmatrix}&&&&\\
&&&\begin{pmatrix}1\\0\end{pmatrix}&&\begin{pmatrix}1\\1\end{pmatrix}&&&\\
&&\begin{pmatrix}2\\0\end{pmatrix}&&\begin{pmatrix}2\\1\end{pmatrix}&&\begin{pmatrix}2\\2\end{pmatrix}&&\\
&\begin{pmatrix}3\\0\end{pmatrix}&&\begin{pmatrix}3\\1\end{pmatrix}&&\begin{pmatrix}3\\2\end{pmatrix}&&\begin{pmatrix}3\\3\end{pmatrix}&\\
\begin{pmatrix}4\\0\end{pmatrix}&&\begin{pmatrix}4\\1\end{pmatrix}&&\begin{pmatrix}4\\2\end{pmatrix}&&\begin{pmatrix}4\\3\end{pmatrix}&&\begin{pmatrix}4\\4\end{pmatrix}\end{matrix}\]
\end{frame}

\begin{frame}[fragile]{Calculadora básica - Combinatoria}
\protect\hypertarget{calculadora-buxe1sica---combinatoria-3}{}
\begin{Shaded}
\begin{Highlighting}[]
\FunctionTok{factorial}\NormalTok{(}\DecValTok{5}\NormalTok{)}
\end{Highlighting}
\end{Shaded}

\begin{verbatim}
[1] 120
\end{verbatim}

\begin{Shaded}
\begin{Highlighting}[]
\FunctionTok{choose}\NormalTok{(}\DecValTok{4}\NormalTok{,}\DecValTok{2}\NormalTok{)}
\end{Highlighting}
\end{Shaded}

\begin{verbatim}
[1] 6
\end{verbatim}

\begin{Shaded}
\begin{Highlighting}[]
\FunctionTok{factorial}\NormalTok{(}\DecValTok{6}\NormalTok{)}
\end{Highlighting}
\end{Shaded}

\begin{verbatim}
[1] 720
\end{verbatim}

\begin{Shaded}
\begin{Highlighting}[]
\FunctionTok{factorial}\NormalTok{(}\DecValTok{5}\NormalTok{)}\SpecialCharTok{*}\DecValTok{6}
\end{Highlighting}
\end{Shaded}

\begin{verbatim}
[1] 720
\end{verbatim}
\end{frame}

\begin{frame}[fragile]{Trigonometría en radianes}
\protect\hypertarget{trigonometruxeda-en-radianes}{}
\begin{longtable}[]{@{}ll@{}}
\toprule()
Código & Función \\
\midrule()
\endhead
\texttt{sin(x)} & \(\sin(x)\) \\
\texttt{cos(x)} & \(\cos(x)\) \\
\texttt{tan(x)} & \(\tan(x)\) \\
\texttt{asin(x)} & \(\arcsin(x)\) \\
\texttt{acos(x)} & \(\arccos(x)\) \\
\texttt{atan(x)} & \(\arctan(x)\) \\
\bottomrule()
\end{longtable}
\end{frame}

\begin{frame}[fragile]{Trigonometría en radianes}
\protect\hypertarget{trigonometruxeda-en-radianes-1}{}
\begin{Shaded}
\begin{Highlighting}[]
\FunctionTok{sin}\NormalTok{(pi}\SpecialCharTok{/}\DecValTok{2}\NormalTok{)}
\end{Highlighting}
\end{Shaded}

\begin{verbatim}
[1] 1
\end{verbatim}

\begin{Shaded}
\begin{Highlighting}[]
\FunctionTok{cos}\NormalTok{(pi)}
\end{Highlighting}
\end{Shaded}

\begin{verbatim}
[1] -1
\end{verbatim}

\begin{Shaded}
\begin{Highlighting}[]
\FunctionTok{tan}\NormalTok{(}\DecValTok{0}\NormalTok{)}
\end{Highlighting}
\end{Shaded}

\begin{verbatim}
[1] 0
\end{verbatim}
\end{frame}

\begin{frame}{Trigonometría en radianes}
\protect\hypertarget{trigonometruxeda-en-radianes-2}{}
\begin{figure}
\centering
\includegraphics{Imgs/trigon.png}
\caption{Circunferencia Goniométrica}
\end{figure}
\end{frame}

\begin{frame}[fragile]{Un pequeño adelanto}
\protect\hypertarget{un-pequeuxf1o-adelanto}{}
\begin{Shaded}
\begin{Highlighting}[]
\NormalTok{x }\OtherTok{=} \FunctionTok{seq}\NormalTok{(}\DecValTok{0}\NormalTok{,}\DecValTok{2}\SpecialCharTok{*}\NormalTok{pi,}\FloatTok{0.1}\NormalTok{)}
\FunctionTok{plot}\NormalTok{(x,}\FunctionTok{sin}\NormalTok{(x),}\AttributeTok{type=}\StringTok{"l"}\NormalTok{,}\AttributeTok{col=}\StringTok{"blue"}\NormalTok{,}\AttributeTok{lwd=}\DecValTok{3}\NormalTok{, }\AttributeTok{xlab=}\FunctionTok{expression}\NormalTok{(x), }\AttributeTok{ylab=}\StringTok{""}\NormalTok{)}
\FunctionTok{lines}\NormalTok{(x,}\FunctionTok{cos}\NormalTok{(x),}\AttributeTok{col=}\StringTok{"green"}\NormalTok{,}\AttributeTok{lwd=}\DecValTok{3}\NormalTok{)}
\FunctionTok{lines}\NormalTok{(x, }\FunctionTok{tan}\NormalTok{(x), }\AttributeTok{col=}\StringTok{"purple"}\NormalTok{,}\AttributeTok{lwd=}\DecValTok{3}\NormalTok{)}
\FunctionTok{legend}\NormalTok{(}\StringTok{"bottomleft"}\NormalTok{,}\AttributeTok{col=}\FunctionTok{c}\NormalTok{(}\StringTok{"blue"}\NormalTok{,}\StringTok{"green"}\NormalTok{,}\StringTok{"purple"}\NormalTok{),}
     \AttributeTok{legend=}\FunctionTok{c}\NormalTok{(}\StringTok{"Seno"}\NormalTok{,}\StringTok{"Coseno"}\NormalTok{, }\StringTok{"Tangente"}\NormalTok{), }\AttributeTok{lwd=}\DecValTok{3}\NormalTok{, }\AttributeTok{bty=}\StringTok{"l"}\NormalTok{)}
\end{Highlighting}
\end{Shaded}

\begin{center}\includegraphics{Tema1_files/figure-beamer/unnamed-chunk-5-1} \end{center}
\end{frame}

\begin{frame}[fragile]{Números en coma flotante}
\protect\hypertarget{nuxfameros-en-coma-flotante}{}
\begin{longtable}[]{@{}
  >{\raggedright\arraybackslash}p{(\columnwidth - 2\tabcolsep) * \real{0.2593}}
  >{\raggedright\arraybackslash}p{(\columnwidth - 2\tabcolsep) * \real{0.7407}}@{}}
\toprule()
\begin{minipage}[b]{\linewidth}\raggedright
Código
\end{minipage} & \begin{minipage}[b]{\linewidth}\raggedright
Función
\end{minipage} \\
\midrule()
\endhead
\texttt{print(x,n)} & Muestra las \(n\) cifras significativa del número
\(x\) \\
\texttt{round(x,n)} & Redondea a \(n\) cifras significativas un
resultado o vector numérico \(x\) \\
\texttt{floor(x)} & \(\lfloor x\rfloor\), parte entera por defecto de
\(x\) \\
\texttt{ceiling(x)} & \(\lceil x\rceil\), parte entera por exceso de
\(x\) \\
\texttt{trunc(x)} & Parte entera de \(x\), eliminando la parte
decimal \\
\bottomrule()
\end{longtable}
\end{frame}

\begin{frame}[fragile]{Números en coma flotante}
\protect\hypertarget{nuxfameros-en-coma-flotante-1}{}
\begin{Shaded}
\begin{Highlighting}[]
\FunctionTok{print}\NormalTok{(pi,}\DecValTok{5}\NormalTok{)}
\end{Highlighting}
\end{Shaded}

\begin{verbatim}
[1] 3.1416
\end{verbatim}

\begin{Shaded}
\begin{Highlighting}[]
\FunctionTok{round}\NormalTok{(pi,}\DecValTok{5}\NormalTok{)}
\end{Highlighting}
\end{Shaded}

\begin{verbatim}
[1] 3.14159
\end{verbatim}

\begin{Shaded}
\begin{Highlighting}[]
\FunctionTok{floor}\NormalTok{(pi)}
\end{Highlighting}
\end{Shaded}

\begin{verbatim}
[1] 3
\end{verbatim}

\begin{Shaded}
\begin{Highlighting}[]
\FunctionTok{ceiling}\NormalTok{(pi)}
\end{Highlighting}
\end{Shaded}

\begin{verbatim}
[1] 4
\end{verbatim}
\end{frame}

\begin{frame}[fragile]{Variables y funciones}
\protect\hypertarget{variables-y-funciones}{}
\begin{itemize}
\tightlist
\item
  \texttt{nombre\_variable\ =\ valor}: define una variable con dicho
  valor
\item
  \texttt{nombre\_función\ =\ function(variable)\{función\}}: define una
  función
\end{itemize}

\begin{Shaded}
\begin{Highlighting}[]
\NormalTok{miVariable }\OtherTok{=} \DecValTok{4}
\NormalTok{doble }\OtherTok{=} \ControlFlowTok{function}\NormalTok{(x)\{}\DecValTok{2}\SpecialCharTok{*}\NormalTok{x\}}
\FunctionTok{doble}\NormalTok{(miVariable)}
\end{Highlighting}
\end{Shaded}

\begin{verbatim}
[1] 8
\end{verbatim}

\begin{Shaded}
\begin{Highlighting}[]
\NormalTok{cuadrado }\OtherTok{=} \ControlFlowTok{function}\NormalTok{(x)\{x}\SpecialCharTok{\^{}}\DecValTok{2}\NormalTok{\}}
\FunctionTok{cuadrado}\NormalTok{(miVariable)}
\end{Highlighting}
\end{Shaded}

\begin{verbatim}
[1] 16
\end{verbatim}
\end{frame}

\begin{frame}[fragile]{Números complejos}
\protect\hypertarget{nuxfameros-complejos}{}
\begin{longtable}[]{@{}
  >{\raggedright\arraybackslash}p{(\columnwidth - 2\tabcolsep) * \real{0.5000}}
  >{\raggedright\arraybackslash}p{(\columnwidth - 2\tabcolsep) * \real{0.5000}}@{}}
\toprule()
\begin{minipage}[b]{\linewidth}\raggedright
Código
\end{minipage} & \begin{minipage}[b]{\linewidth}\raggedright
Función
\end{minipage} \\
\midrule()
\endhead
\texttt{a+bi} &
\href{https://es.wikipedia.org/wiki/Número_complejo}{Número complejo} \\
\texttt{complex(real=...,imaginary=...)} & Número complejo en forma
binómica \\
\texttt{complex(modulus=...,argument=...)} & Número complejo en forma
polar \\
\bottomrule()
\end{longtable}
\end{frame}

\begin{frame}[fragile]{Números complejos}
\protect\hypertarget{nuxfameros-complejos-1}{}
\begin{longtable}[]{@{}ll@{}}
\toprule()
Código & Función \\
\midrule()
\endhead
\texttt{sqrt(as.complex(-x))} & \(\sqrt{-x}\) \\
\texttt{Re(x)} & Parte real de \(x\) \\
\texttt{Im(x)} & Parte imaginaria de \(x\) \\
\texttt{Mod(x)} & Módulo de \(x\) \\
\texttt{Arg(x)} & Argumento de \(x\) \\
\texttt{Conj(x)} & Conjugado de \(x\) \\
\bottomrule()
\end{longtable}
\end{frame}

\begin{frame}[fragile]{Números complejos}
\protect\hypertarget{nuxfameros-complejos-2}{}
\begin{Shaded}
\begin{Highlighting}[]
\NormalTok{z }\OtherTok{=} \DecValTok{2}\SpecialCharTok{+}\NormalTok{3i}
\NormalTok{z2 }\OtherTok{=} \FunctionTok{complex}\NormalTok{(}\AttributeTok{real =} \DecValTok{2}\NormalTok{, }\AttributeTok{imaginary =} \SpecialCharTok{{-}}\DecValTok{3}\NormalTok{)}
\FunctionTok{Re}\NormalTok{(z)}
\end{Highlighting}
\end{Shaded}

\begin{verbatim}
[1] 2
\end{verbatim}

\begin{Shaded}
\begin{Highlighting}[]
\FunctionTok{Im}\NormalTok{(z)}
\end{Highlighting}
\end{Shaded}

\begin{verbatim}
[1] 3
\end{verbatim}

\begin{Shaded}
\begin{Highlighting}[]
\FunctionTok{Conj}\NormalTok{(z2)}
\end{Highlighting}
\end{Shaded}

\begin{verbatim}
[1] 2+3i
\end{verbatim}
\end{frame}

\begin{frame}{Números complejos}
\protect\hypertarget{nuxfameros-complejos-3}{}
\includegraphics{Imgs/complex.png}
\end{frame}

\end{document}
